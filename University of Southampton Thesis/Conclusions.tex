\chapter{Conclusions} \label{Chapter:Conclusions}

In this thesis efforts to combine microfluidics with micro-NMR are described.

Enabling microfluidic experiments to be compatible with NMR offers advantages
of label-free, non-destructive and non-invasive analysis which can easily
be combined with existing methods of investigation and used to
enhance the information that can be gleaned form a system.

In chapter \ref{Chapter:Droplets} a device capable of generating microfludic
droplets and performing high-resolution NMR is presented. They key to doing this
is matching the susceptibilities of the fluids and materials used. This is done
by cutting shim structures filled with air around the sample chamber and by
doping the aqueous phase with chelated $\ce{Eu^3+}$ ions to bring them inline
with the susceptibility of the oil phase, in this case cyclohexane. The
precise location and size of the shim structures was simulated and the
exact concentration of europium measured in order to collect a high
resolution spectrum of droplets containing glucose, comparable to the spectrum
obtained by pure phase glucose and could pave the way for droplet NMR to be used
in future.

An NMR experiment scaled down to micro-NMR is described in chapter \ref{Chapter:Parahydrogen},
where parahydrogen induced polarisation (PHIP) reactions where performed on a device
capable of bubble free hydrogenation. The PASADENA reaction presented in the chapter
leads to enhancement factors of ~1800 when compared with a using normal hydrogen. The microfluidic
aspect of the device allows the production of the hyperpolarised species in continuous flow, this
stability allows for the collection of 2D hyperpolarised spectra. This device could provide a
way to produce hyperpolarised metabolites continously and either introduce them to  cell culture
or other living tissue \textit{in situ} or for injection and \textit{in vitro} imaging.

Finally, chapter \ref{Chapter:Peristaltics} describes the design and manufacture of a
peristaltic pump device capable of exhange and mixing of fluids within the device in
a high field NMR magnet. The design couples together a PDMS mebrane and structures cut into
the device to form valves which are individually addressable to perform the pumping and mixing
routines, whilst keeping dead volume to a minimum. Operation of this device \textit{in situ} is presented,
with spectra shown, demonstrating the exchange and mixing of two fluids in the device.

These three examples of the successful combination microfluidics with high-field high resolution micro-NMR
spectroscopy, show how the two fields can be used to enhance and compliment one another without compromise. The
novelty and function of either part, microfluidic or NMR, are not sacrificed for the other. They make
for exciting prospects, the droplets could be used to investigate and track oxygen concentrations in living
tissue culture, the continuous PHIP device could provide new insights into the kinetics of the PASADENA
reaction reported, as well as the formation of metabolites directly that could be introduced to a cell culture and
the metabolomics tracked using NMR. The pump can be combined with microfluidic experiments that are mass limited
for example, ligand binding studies of proteins or could be used in cell culture experiments to perfuse the
culture with media and provide oxygen and nutrients whilst enabling the observation of the culture by NMR.

Overall, microfluidic NMR shows promise in challenging new problems and offers exciting possibilities
when combined with existing techniques.
