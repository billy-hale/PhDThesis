% !TEX root = droplet-shim-manuscript.tex

%\subsection*{Introduction}

NMR spectroscopy is one of the most important analytical tools
available to chemistry, biochemistry, and the life sciences.
Due to its linearity and generality, it is particularly suited
to quantify metabolic compounds in biological systems.\cite{Aranibar:2011dc,Wishart:2008ga}
At the same time, microfluidic
technology is rapidly evolving, enabling numerous novel applications in
chemistry and the life sciences. Droplet
microfluidics is based on the separation of samples into small droplets
suspended in an inert transport fluid (often a fluorinated oil).
\cite{Thorsen:2001gt,Anna:2003ci,Gunther:2006vd,Garstecki:2006tn}
In this way, samples can be manipulated freely in the LoC system,
and problems due to viscous dispersion and cross-contamination
are avoided. For example, several groups have reported encapsulation
of cells into individual droplets.\cite{Lagus:2013bxb,Mazutis:2013ig}  Droplets can
efficiently be sorted according to numerous chemical and
biochemical criteria by the help of suitable fluorescent markers.
As a result, droplet microfluidic systems are increasingly finding applications
in chemistry\cite{Theberge:2012iq} and the
life sciences.\cite{Zhu:2013er,Mazutis:2013ig} In this paper, we explore
the possibility to obtain high-resolution NMR spectra from
small volumes of droplet emulsions on a chip.
Integration of high-resolution NMR spectroscopy with microfluidic systems is
challenging for a number of reasons.
On the one hand, small sample volumes place stringent demands on detector
sensitivity.\cite{Badilita:2011td,Zalesskiy:2014hi}
This has recently been addressed with the design of
highly efficient planar NMR microcoils \cite{Spengler:2016km} and
transmission line resonators.\cite{Finch:2016gv}
Another challenge is the preservation of high spectral
resolution, which depends on a highly homogeneous magnetic field
over the sample volume. Differences in magnetic susceptibility
between the materials used for the microfluidic chip
and the sample fluid, as well as the materials and geometry
of the probe assembly, lead to a demagnetising field
that varies continously over the sample volume. Typical diamagnetic
volume susceptibilties range from about
$-11$~ppm to about $-5$~ppm (in SI units);\cite{Kuchel:2003ip,Durrant:2003kv}
differences of the order of several ppm are therefore commonplace.
Unmanaged, they lead to
broadening of NMR spectral lines over a ppm or more, which
corresponds to severe loss of resolution in \ce{^1H} liquid
state NMR.

In emulsions, susceptibility differences between the oil and aqueous phases
lead to similar line broadening.\cite{Kuchel:2003ip} NMR spectroscopy is extensively
used to characterise emulsion droplet size distributions using pulsed
field gradient methods.\cite{VANDENENDEN:1990ck,FOUREL:1994jv,
Hollingsworth:2004iy,Hindmarsh:2005en,Johns:2009ib,Bernewitz:2011km,Lingwood:2012je}
These methods do not require spectral resolution of individual
compounds other than the two solvents, and are therefore
unaffected by the susceptibility broadening. By contrast, high-resolution
NMR spectroscopy, with sufficient resolution to distinghuish multiple
compounds present in either of the two phases,
requires careful mitigation of the susceptibility
differences.

We have recently shown that it is possible to compensate for
susceptibility differences between microfluidic chips and the sample
fluid by incorporating appropriately shaped shim structures
into the chip design.\cite{Ryan:2014hl} These structures are filled with air,
and are shaped in such a way as to cause demagnetising fields
that are equal and opposite to those caused by the sample/chip interface.


Managing susceptibility differences for an emulsion of droplets on a
microfluidic chip adds additional complexity, since three different materials
are now involved: the chip, the continuous phase, and the droplet phase,
all with different susceptibilities.
In the following, we show that this can be mitigated in a two-step approach,
which is based on the observation that most organic solvents in use
as continuous phases for droplet microfluidics are less diamagnetic than
water.
First, the susceptibility difference between the chip and the continuous phase
are compensated by shimming structures that are added to the
chip design. Then, the susceptibility of the aqueous droplet phase is matched
to that of the continuous phase by adding a paramagnetic solute.

\ce{Eu^{3+}} complexes are paramagnetic, and are frequently used as
shift agents in NMR spectroscopy. Unlike other lanthanide ions such
as \ce{Gd^{3+}} or \ce{Ho^{3+}}, \ce{Eu^{3+}} has only a minimal
effect on nuclear magnetic relaxation times.\cite{Peters:1996bj} Addition of
millimolar quantities of \ce{Eu^{3+}} to aqueous solutions
therefore does not cause significant relaxation line broadening, but changes
the bulk magnetic susceptibility of the solution proportionally
to the \ce{Eu^{3+}} concentration. It is therefore possible
to adjust the susceptibility difference in a droplet emulsion
by adding a \ce{Eu^{3+}} complex that selectively dissolves in (or at least
strongly partitions to) the aqueous phase.


\begin{figure}
  \begin{center}
    \includegraphics[width=\columnwidth]{chip-design}
  \end{center}
  \caption{Droplet chip design (left) and detail micrograph of the sample chamber
  area filled with droplets (right). Some droplets are also visible in the
  entrance and exit channels.}
  \label{fig:chip-design}
\end{figure}

\begin{table}
  \begin{center}
    \caption{Bulk magnetic susceptibilities
    \label{tab:suscept}}
    \begin{tabular}{lcc}\hline\hline
      \emph{Compound} & $\chi_V/10^{-6}$ (SI) & Ref \\ \hline
      water           & $-9.05$               &    \cite{Rumble:2017tp}  \\
      cyclohexane     & $-7.640$              &    \cite{Rumble:2017tp} \\
      PMMA            & $-9.01$               &    \cite{Wapler:2014es}\\
      Air             & $+0.36$               &    \cite{Bakker:2006eea} \\ \hline\hline
    \end{tabular}
  \end{center}
\end{table}

\begin{figure}
  \begin{center}
    \includegraphics[width=\columnwidth]{fcc-FEM}
  \end{center}
  \caption{A: Finite element simulation of magnetic field distribution in droplets.
      $z$-component of the reduced magnetic field $H_\text{red}$ in an isolated spherical droplet
      and B: in a face-centred cubic arrangment of droplets; C: FEM mesh used
      to calculate the result shown in B; D: histograms of the $z$-component
      of the reduced magnetic field in the continuous (orange) and in the droplet (blue) phase
      in the FCC arrangement.
    }
  \label{fig:FEM-fcc}
\end{figure}

In the present work, we use the diethyl-triamine pentaacetate (DTPA) complex
of \ce{Eu^{3+}}, \ce{Eu[DTPA]^{2-}}. As an ion species, it is readily soluble
in aqueous media, while exhibiting only negilgible solubility in apolar organic
solvents. Microfluidic chips are fabricated from poly methyl methacrylate (PMMA).
By a fortunate coincidence, the suceptibilities of PMMA and
water are very close to
each other (Table \ref{tab:suscept}).  NMR lines
in microfluidic devices made from PMMA are therefore narrow
if  aqueous
samples are used, provided that the boundaries of the chip and the environment
are either aligned with the external magnetic field, or are kept sufficiently
remote from the detection area.
 By contrast, most organic solvents are
considerably less diamagnetic than water, as exemplified by the
case of cyclohexane, which has been used in the present study.

In the remainder of this paper, we first use
finite element calculations to estimate
the NMR line widths expected in a droplet emulsion
depending on the susceptibility mismatch.
The results are then compared to experimental
line widths obtained with varying concentrations
of \ce{Eu[DTPA]^{2-}} in the aqueous phase. Finally,
we show that narrow NMR lines
can be obtained by
combining structural shimming \cite{Ryan:2014hl} with
susceptibility matching, and demonstrate that this
approach can be used with advanced
NMR techniques such as heteronuclear
single-quantum correlation (HSQC) spectroscopy.
The chip used in this work is shown in \fig{fig:chip-design}. It consists of a sample
chamber in the centre of the chip, which is designed to line up with
the sensitive area of a transmission-line
micro-NMR detector,\cite{Finch:2016gv} and a convergent flow droplet
generator. The aqueous phase droplet phase and the continuous
phase are fed into the two ports at the top. Droplets are
formed and transported downstream into the sample chamber.
 The chamber is surrounded by two shim structures, which are half-moon
 shaped cutouts filled with air. They have been designed to compensate
 for the difference in susceptibility between the chip material (PMMA)
 and the oil phase (cyclohexane). The operation of the chip is shown
 on the right side of \fig{fig:chip-design}; droplets of about 140~$\mu$m diameter are formed and
 fill the sample chamber.


%\subsection*{Materials and Methods}
Microfluidic chips of the design shown in \fig{fig:chip-design}
were fabricated from PMMA sheet material by
laser cutting, and subsequent bonding of layers with a plasticiser
under heat and pressure.\cite{Yilmaz:2016fx} The chips consist of a
top and bottom layer of 200~$\mu$m thickness each, and a middle layer of
500~$\mu$m. Fluid channels upstream from the flow-focussing droplet
generator were scored into the middle layer at low laser power to a depth
of about 100~$\mu$m. Downstream from the droplet generator, the channels and
the sample chamber were cut through the 500 $\mu$m middle layer by increased
laser power, as were the shimming structure. The chips were connected to
a pair of LabSmith uProcess syringe pumps for droplet generation.
A flow rate of 15~$\mu$l/min was typically used for the continuous
phase and 1.5~$\mu$l/min for the aqueous droplet phase.
The continuous phase consisted of cyclohexane (Sigma-Aldrich)
with 0.5\% w/v of span-65 (sorbitan
tristearate, Sigma-Aldrich) as a surfactant to ensure droplet stability.
Steady state conditions were ensured by letting the droplet generation run until
the volume inside the chip had been exchanged at least five times. The
chip was then disconnected from the syringe pumps, and the connection
points sealed prior to insertion of the chip into the NMR probe.

NMR measurements were carried out on a Bruker AVANCE III
spectrometer equipped with an Oxford wide bore magnet operating at 7.05 Tesla,
corresponding to a \ce{^1H} Larmor frequency of 300 MHz. A home-built NMR probe
based on a transmission-line detector was used.\cite{Finch:2016gv}
It accommodates microfluidic chips of the shape shown in \fig{fig:chip-design}.
In the present work, the probe was doubly
tuned to allow irradiation both at 300 MHz for \ce{^1H} and at 75 MHz for
\ce{^{13}C}. Details of the electronic and mechanical design of the
probe are given in \cite{Finch:2017vb}.

NMR spectra were obtained at RF nutation frequencies of 66 kHz for
\ce{^1H} and 28 kHz for \ce{^{13}C}, corresponding to 90 degree pulse lengths of
3.8 $\mu$s and 9 $\mu$s, respectively. Heteronuclear
single-quantum correlation (HSQC) spectra were obtained using
GARP heteronuclear decoupling during the acquisition period with a
13C nutation frequency of 3.5~kHz.
100 $t_1$ increments were acquired in 16 scans over a total experiment time of
60 min. NMR data were acquired using Bruker spectrometer software (TopSpin 2.0),
and were processed using home-built scripts written in Mathematica.
20~mM tetramethyl silane (TMS, Sigma Aldrich) was added to the continuous phase
as a chemical shift standard.

\ce{Eu[DTPA]^{2-}} solutions were prepared  from a $100\pm0.25$~mM
stock solution, which was prepared by adding 0.258 g of \ce{EuCl_3}
(Sigma Aldrich) to 4 ml of deinonised water (Sigma Aldrich). Separately,
0.393 g of diethylenetriaminepentaacetic acid (DTPA, Sigma Aldrich)
were dissolved in 0.6 ml of 10 M NaOH and 3.4 ml of DI water.
The \ce{EuCl_3} solution was slowly added to the DTPA solution under
stirring over a period of 5 min. The pH was then measured,
and 0.46 ml of 1M NaOH were gradually added to adjust the solution to pH 7.

Finite element calculations were carried out using COMSOL 5.2a with the "magnetic
fields, no currents" (mfnc) physics module.

%\subsection*{Results and Discussion}

\begin{figure}
  \begin{center}
    \includegraphics[width=0.7\columnwidth]{predicted-spectra}
  \end{center}
  \caption{Predicted 1H NMR line shapes of water (left) and cyclohexane (right) of a
  water in cyclohexane emulsion as a function of {Eu[DTPA]} concentration
  in the aqueous phase.
  }
  \label{fig:predicted-spectra}
\end{figure}


\begin{figure}
  \begin{center}
    \includegraphics[width=0.7\columnwidth]{droplet-spectra}
  \end{center}
  \caption{1H NMR line shapes of water (left) and cyclohexane (right) of a
  water in cyclohexane emulsion as a function of {Eu[DTPA]} concentration
  in the aqueous phase.
  }
  \label{fig:droplet-spectra}
\end{figure}




While it is possible to predict the magnetic field distribution in a system of
multiple phases with differing susceptibilities by solving the magnetostatic
equation, this requires precise geometric information on the arragement of the
two phases. In the case of an emulsion, the arrangement of the droplets is not
regular. However, at high droplet densities, it can be
expected to approximate  a dense packing of spheres. In order to obtain a
semi-quantitative prediction, we have computed the demagnetising field in
face-centred cubic  (FCC) and simple cubic (SC) lattices of diamagnetic spheres;
the results are shown in  \fig{fig:FEM-fcc}. A single unit cell containing
one (SC) or
two (FCC) independent spheres was meshed under periodic boundary conditions in all
directions (\fig{fig:FEM-fcc}C).
As is well known, the demagnetising field inside an isolated diamagnetic sphere
is homogeneous, while the field outside of the
sphere is that of a magnetic
point dipole located at the sphere's centre. This situation is approximated
in a lattice if the lattice constant is much larger than the sphere
diameter. The computed demagnetising field
of a small sphere in an SC lattice is shown in \fig{fig:FEM-fcc}A.
The contour levels display the $z$-component of the local demagnetising field
normalised by the background $B_0$ field and the susceptibility difference
$\Delta\chi= \chi_\text{sphere}-\chi_\text{continuous}$. The field is homogeneous
inside the sphere, and a spatially varying demagnetising field only
exists in the continuous phase.
By contrast, in a
densely packed face-centered cubic lattice the field is no
longer homogeneous inside the spheres (\fig{fig:FEM-fcc}B). The FCC
lattice approximates the geometry of a dense microemulsion of homogenous
water-in-oil droplets.  \fig{fig:FEM-fcc}D shows the histograms of
the $z$-components of
the demagnetisig field in the continuous and droplet phases of the FCC
lattice, respectively.

The NMR spectra expected from an ideal emulsion of the same geometry can
be predicted from these histograms
(neglecting no broadening contributions from the sample container).
The magnetic field relevant for nuclear Larmor precession, often referred
to as the "external" field\cite{Levitt:1996tg} $\mathbf{B}_\text{ext}$  is
given by\cite{Ryan:2014hl}
\[
\mathbf{B}_\text{ext}(\mathbf{r})
= B_0 (1+\frac{\chi_s}{3}) \mathbf{e}_z  - {\mu_0} \nabla U_d(\mathbf{r}),
\]
where $B_0$ is the magnitude of the external field, $\chi_s$ is the local
magnetic susceptibility, and $U_d(\mathbf{r})$ is the scalar magnetic potential
of the demagnetising field. The volume susceptibility of a solution containing a
paramagnetic species at low concentration $c_p$ is
\[
    \chi_s \approx \chi_0 + c_p\,\zeta_P,
\]
where $\chi_0$ is the volume susceptibility of the pure solvent,
and $\zeta_P$ is the
molar susceptibility of the paramagnetic species. $\zeta_P$ depends
slightly on the molecular environment. For example, values of
$5.86\cdot 10^{-5}\;\mathrm{l/\text{Mol}}$,
$5.68\cdot 10^{-5}\;\mathrm{l/\text{Mol}}$, and
$6.14\cdot 10^{-5}\;\mathrm{l/\text{Mol}}$ have been measured at 300K for
\ce{Eu_2O_3}, \ce{EuF_3}, and \ce{EuBO_3}, respectively.\cite{Takikawa:2010iw}
To our knowledge, the precise molar susceptibility of \ce{Eu[DTPA]^{2-}} in
aqueous solution has not been measured to date, but it is likely to be
similar to the above values.

In this work, we have found a \ce{Eu[DTPA]^{2-}} concentration of $c_p=31.0$~mM
to produce the sharpest NMR resonance lines in an emulsion of aqueous droplets
in cyclohexane. Using the susceptibilities given in
Table \ref{tab:suscept}, this leads to molar susceptibility for \ce{Eu[DTPA]^{2-}}
of $4.56\cdot 10^{-5}\;\mathrm{l/\text{Mol}}$. However, we regard this
value as preliminary, as it may be affected by errors in the \ce{Eu[DTPA]^{2-}}
stock solution concentration. Accepting it for the time being,
and using the histograms shown in \fig{fig:FEM-fcc}D,
we can predict emulsion NMR
spectra as a function of \ce{Eu[DTPA]^{2-}} concentration in the aqueous phase,
as shown in \fig{fig:predicted-spectra}. Both peaks are broadened by the
susceptibility mismatch, with the narrowest line widths predicted near the
optimum concentration. The \ce{H_2O} (droplet) peak is less affected than
the cyclohexane (continuous phase) peak, as expected from the histograms
in \fig{fig:FEM-fcc}D.

\begin{figure}
  \begin{center}
    \includegraphics[width=0.8\columnwidth]{linewidth-plot}
  \end{center}
  \caption{Observed line widths of water (brown circles),
    cyclohexane (red squares), and the TMS
    reference (blue circles) in microfluidic droplet emulsions
    as a function of the \ce{Eu[DTPA]^{2-}}-concentration in the aqueous phase.
    The widths of all three lines are minimal
    at the matched concentration of 31.0~mM.}
    \label{fig:linewidths}
\end{figure}

\begin{figure}
  \begin{center}
    \includegraphics[width=0.8\columnwidth]{proline-HSQC-trace}
  \end{center}
  \caption{\ce{^1H}-\ce{^{13}C} heteronuclear single-quantum correlation (HSQC) spectrum of 2 $\mu$l
  of 20 mM proline (u-\ce{^{13}C},\ce{^{15}N}) in
    a water-in-cyclohexane emulsion with 31.0~mM \ce{Eu[DTPA]^{2-}} in the aqueous phase,
    obtained with the microfluidic chip of \fig{fig:chip-design}.
    Droplets are about 140 $\mu$m in size. 200 $t_1$ increments were obtained,
    total experiment duration 1h.}
    \label{fig:proline-HSQC}
\end{figure}




\begin{figure}
\begin{center}
	\includegraphics[width=0.8\columnwidth]{mu1y11-dnmr-fi-180414-001-glucose-wsup}
\end{center}
\caption{Spectra of 200~mM Glucose in \ce{H_2O} obtained from microfluidic droplet emulsions in
	cyclohexane. 1: Aqueous phase contains $c_0=23.75\pm0.25$~mM \ce{Eu[DTPA]^{2-}}. Spectra 2-7
	have been obtained by gradual dilution of the aqueous phase with small amounts of DI water.
	2: $\ln c/c0 = -0.5\%$; 3: $\ln c/c0 = -0.75\%$; $4: \ln c/c0 = -0.875\%$;
	$5: \ln c/c0 = -1.0\%$; $6: \ln c/c0 = -1.125\%$; $7: \ln c/c0 = -1.25\%$.}
\label{fig:glucose-dilution}
\end{figure}


\fig{fig:droplet-spectra} shows experimental \ce{^1H} NMR spectra of
a water droplet/cyclo\-hexane emulsion obtained with the chip shown in
\fig{fig:chip-design}, with droplets of approximately 150 $\mu$m in diameter,
at various concentrations of \ce{Eu[DTPA]^{2-}} in the droplet phase as
indicated. The spectra show a water peak near 4.7~ppm, and a
cyclohexane peak near 1.4~ppm. Both peak line widths are affected by the
\ce{Eu[DTPA]^{2-}} concentration, reaching a minimum width at the optimal
concentration of 31.0~mM. The width of the cyclohexane peak is in quantiative
agreement with the simulation predictions, including some details like
the skewness of the peak, which appears on the right (upfield) side
of the peak at at 20mM, and on the left (downfield) side above
the optimal concentration at 35~mM. The water peak, which
stems from the droplet phase, exhibits similar broadening behaviour. This
is in contrast to the simulations, which predicted a much sharper line shape
for the droplet phase. There may be several reasons for this discrepancy.
On the one hand, the simulations had assumed a highly regular arrangement
of perfectly uniform droplets. Clearly, this is a crude approximation.
Another influence may stem from the interface between the microfluidic chip
and the emulsion. Away from the optimum \ce{Eu[DTPA]^{2-}} concentration, the
averaged susceptibility of the emulsion differs from that of cyclohexane,
for which the shimming structures in the chip had been designed. This
causes additional line broadening, affecting both phases.

The observed widths of the NMR signals from cyclohexane, water, and the
chemical shift standard TMS (which dissolves only in the continuous phase)
are summarised in \fig{fig:linewidths}. All three exhibit a narrow
minimum at 31~mM \ce{Eu[DTPA]^{2-}} in the aqueous phase. The water and
cyclohexane minimum peak widths are 3.2~Hz and 3.5~Hz, respectively, while
the TMS signal narrows to 2.7~Hz. For comparison, the best resolution
that has been reached with the same NMR probe is 1.76~Hz for a homogeneous
solution of 150~mM sodium acetate in \ce{H_2O}.\cite{Finch:2016gv}
Resolutions in this vicinity are sufficient for proton NMR spectroscopy.
This is demonstrated by the heteronuclear single-quantum correlation spectrum
of a droplet emulsion with 20~mM universally \ce{^{13}C}, \ce{^{15}N}-labelled
proline, shown in \fig{fig:proline-HSQC}. Peaks for all protonated
carbon sites in the molecule are clearly visible and resolved. The water signal
has been suppressed by pre-saturation, while there is a strong peak from
natural abundance \ce{^{13}C} in cyclohexane at $\delta^{13}C=29$~ppm,
$\delta^{1}H=1.45$~ppm. The ripples emanating from this signal in the \ce{^{13}C}
domain are artifacts due to minor instabilities of the spectrometer between
different $t_1$ increments.
The quality of this spectrum is
essentially the same as one acquired in a pure aqueous solution. This clearly
demonstrates the feasibility of high-resolution NMR spectroscopy in microfluidic
droplet emulsions.

In summary, we have shown that the susceptibility differences between the chip,
the aqueous phase, and the oil phase in a microfluidic droplet system can
be successfully mitigated by a combination of structural shimming and
doping of the less diamagnetic of the liquid phases with a europium compound.
The ultimate resolution achieved is only slightly inferior to what has been
demonstrated in homogeneous solutions on a microfluidic chip. While the droplets
in the present work were about 150~$\mu$m in diameter, many lab-on-chip systems
operate with significantly smaller droplets, typically in the range of
$10\dots 20$~$\mu$m. Work to demonstrate the feasibility and robustness of
this approach for such small droplets is currently underway in our laboratory,
and will be reported on a future occasion.

This work has been supported by the 7th EU Framework programme through a
Marie Curie Career Integration Fellowship to MU, and by the Horizon 2020 FETOPEN
project TISuMR. The authors are grateful to Visvaldis Buns and Ali Yilmaz for help
with manufacturing of the microfluidic chip.

\scriptsize{
\bibliography{literature} %You need to replace "rsc" on this line with the name of your .bib file
\bibliographystyle{rsc} } %the RSC's .bst file
